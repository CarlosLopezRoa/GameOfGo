\documentclass[a4paper,10pt,twocolumn]{article}
%\usepackage{xeCJK}
%\setCJKmainfont{SimSun}
\usepackage{CJKutf8}
\usepackage{cite}
\usepackage{listings}
\usepackage[T1]{fontenc}
\usepackage{amsmath}
\usepackage{amssymb}
\usepackage{graphicx}
\usepackage[]{circuitikz}
\setlength{\oddsidemargin}{0in}
\setlength{\topmargin}{-.8in}
\setlength{\textheight}{9.7in} \setlength{\textwidth}{6.5in}
\usepackage{color,hyperref}
\definecolor{darkblue}{rgb}{0.0,0.0,0.3}
\hypersetup{colorlinks,breaklinks,
            linkcolor=darkblue,urlcolor=darkblue,
            anchorcolor=darkblue,citecolor=darkblue}
\providecommand*\url[1]{\href{#1}{#1}}
\renewcommand*\url[1]{\href{#1}{\texttt{#1}}}

\newcommand{\bm}[1]{\boldsymbol{#1}}
\newcommand{\bh}[1]{\boldsymbol{\hat{#1}}}
\newcommand{\bt}[1]{\boldsymbol{\tilde{#1}}}
\newcommand{\bbar}[1]{\boldsymbol{\bar{#1}}}
\newcommand{\mbf}[1]{\ensuremath{\mathbf{#1}}}
\newcommand{\ode}[2]{\ensuremath{\frac{\mathrm{d} #1}{\mathrm{d} #2}}}
\newcommand{\odet}[2]{\ensuremath{\tfrac{\mathrm{d} #1}{\mathrm{d} #2}}}
\newcommand{\oden}[3]{\ensuremath{\frac{\mathrm{d}^#3 #1}{\mathrm{d} #2^#3}}}
\newcommand{\pde}[2]{\ensuremath{\frac{\partial #1}{\partial #2}}}
\newcommand{\pdet}[2]{\ensuremath{\tfrac{\partial #1}{\partial #2}}}
\newcommand{\pden}[3]{\ensuremath{\frac{\partial^{#3} #1}{\partial
      #2^{#3}}}}
\newcommand{\sub}[1]{\ensuremath{_{\rm{#1}}}}
\newcommand{\arriba}[1]{\ensuremath{^{\rm{#1}}}}
%
\newcommand{\N}{\ensuremath{\mathbb{N}}}
\newcommand{\R}{\ensuremath{\mathbb{R}}}
\newcommand{\C}{\ensuremath{\mathbb{C}}}
\newcommand{\ee}[1]{\ensuremath{\mathrm{e}^{#1}}}
\newcommand{\hdos}{\ensuremath{\mathrm{H}_2}}
\newcommand{\COdos}{\ensuremath{\mathrm{CO}_2}}
\newcommand{\ATP}{\ensuremath{\mathrm{ATP}}}

\newcommand{\dt}{\ensuremath{\mathrm{d}t}}
\newcommand{\dtau}{\ensuremath{\mathrm{d}\tau}}
\newcommand{\DV}{\ensuremath{\Delta V}}
\DeclareMathOperator{\Li}{\mathcal {L}^{-1}}
\DeclareMathOperator{\Lin}{\mathcal {L}^{-1}}
\DeclareMathOperator{\sinc}{\text{sinc}}
\DeclareMathOperator{\sign}{\mathrm{sign}}
\newtheorem{remark}{Remark}
%\usepackage[utf8x]{inputenc}
%%
\title{围棋 \\ The Game of Go\\ \small Modeling Complex Systems\\ DMKM}
\author{Carlos López Roa\\ \href{mailto:me@mr3m.me}{me@mr3m.me}}
\date{\today}
\pdfinfo{%
  /Title    ()
  /Author   (CLR)
  /Creator  ()
  /Producer ()
  /Subject  ()
  /Keywords ()
}
\begin{document}
\begin{CJK*}{UTF8}{gbsn}
\maketitle
%%% Content
\begin{abstract}
https://youtu.be/Wvm0ZgCsm1E
https://github.com/mr3m/GameOfGo
\end{abstract}

%\tableofcontents

\section{Introduction}
The game of Go, also known as \emph{Wéiqí: 围棋} in Chinese, which means literally \emph{encircling game}, is a two-player board game in which the aim is to surround more territory that the opponent \cite{Kunkle2002}.

It was originated in ancient China more than 2,500 years ago. It was considered one of the four essential arts\footnote{The four arts ({siyi: 四艺}): To play the guqin, a stringed instrument ({Qín:  琴}), the strategy game of Go ({Qí:  棋}), Chinese calligraphy ({Shu:  书}), Chinese painting ({Huà:  画}) } of a cultured Chinese scholar.

Despite it's relative simple rules, the relative complexity of Go with respect to Western chess is far more superior ($10^{761}$ compared to $10^{120}$ possible games)

Precisely because of this great complexity is that, different from chess which was \emph{conquered} by IBM's \textsc{Deep blue} in 1996 agains't world's Grand Master \textsc{Gary Kaspárov}, no equivalent conquest has been achieved by computer go until recent victory of Google's \textsc{Alpha Go} agains't the European Go Champion \cite{Silver2016a}. 

In this study we took a multiagent system approach to the problem of computer Go, first, a description of the implementation is made, after some tests were carried out and described at the end some conclusions and future work are drawn.

%\subsection{State of the art}

\subsection{Game Mechanics}
The two players alternately place black and white playing pieces, called \emph{stones}, on the vacant places of a board with a 19×19 grid of lines. 

Each stone is said to have 4 \emph{liberties} ({Qì:  气})  when the four orthogonal-adjacent points are empty, this stone loses each of it's liberties whenever a stone is placed in this points. If the recently placed stone is of the opposite color, then the liberty is just lost, however, if the recently placed stone is of the same color, the individual liberty is lost and a \emph{group} liberty is created, composed of the sum of the liberties of these two stones. Generalizing the previous principle, one can form huge groups of stones of the same color, which have a collective liberty. The particular form of this groups is essential part of the strategy of the game. 

Once placed on the board, stones may not be moved, but stones may be removed from the board if captured. A stone (or group of stones) it's captured by the opponent when all the liberties are suppressed.

The two players place stones alternately until they reach a point at which neither player wishes to make another move. When a game concludes, the territory is counted along with captured stones to determine the winner. 



\section{Development}
A box world was set up in NetLogo, to represent the board game, that is, with no periodicity in the edges.  

\section{Results}

\section{Conclusions}

\section{Future Work}



\bibliography{/Users/Poincare/Dropbox/Tex/library.bib}
\bibliographystyle{siam}
\end{CJK*}
\end{document}